\begin{section}{Dificuldades e Solu��es}
\begin{itemize}
\item I was thinking about use the LAMP server to store data transferred from
different kinds of remote controllers, so we could have a database fulfilled
with information that would allow us to control a great range of TV devices.
However I don't know how hard is to access the database from the Julius source.
I mean, the default way to access a database is from PHP Code. I've seen Java
for web that can incorporate SQL commands into the code, json and stuff like
that, but I think to implement a code in C/C++ to push and pull data to a
database is a monkey job. If some of you can try to deal with databases managed
by C++ I would be glad.

\item There's no sound output on BBB. The default audio device, as mentioned in
some foruns on the web, is through the HDMI connector. The same forum has said
that we can disable the HDMI as audio device by changing some kernel parameters,
then the second device would become the main one. In this case, the USB would be
the principal sound card. The easist way, then, would be attach an USB speaker
to the connector, because the TTS would automatically output the result to that
port. My point about the USB speaker:
\begin{itemize}
\item It must be low power consuming
\item Its size must be limited. If the project was to build a thermistor plugged
on the wall, we could not put a big speaker there. The circuit must be smaller
than the BBB itself.
\end{itemize}
It would be nice if we could build an amplifier and put the sinthesized audio  
through a BBB GPIO, but...
\begin{itemize}
\item I don't know how to manipulate the TTS output to put in that pin
\item Even If I could do that, It's hard to play a sound with PWM. A chunck of
code, size dependent of the sampling rate, should be put in that GPIO in a slot
of time. And I think it's not as simple as this description.
\item The arduino codes I've found that play sound through a pin have the
craziest codes I've ever seen. Almost everything is manually done.
\end{itemize}

\item There's an arduino code that is capable to ``hack'' the information coming
from any remote control. It acts as the receiver that we find in front of any TV
device by using an IR sensor. Besides that, we could read the datasheet of the
TV we're gonna use.
\end{itemize}
\end{section}
%%% EOF %%%

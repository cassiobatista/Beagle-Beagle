\begin{section}{Trabalhos Futuros}
% http://www.nngroup.com/articles/remote-control-anarchy/
De acordo com o relato dispon�vel em~\cite{nielsen04}: ``Dado que o meu home
theater � modesto, ele requer que eu consiga manejar APENAS 6 controles remotos
para a simples tarefa de assistir a um filme''. Seria marveolous se houvesse um
controle remoto universal que permitisse acesso � TODOS os aparelhos do 
ambiente residencial, mesmo os que est�o em c�modos diferentes do que eu estou
agora. Seria mais maravilhoso que esse controle estivesse sempre com voc�. E que
pudesse us�-lo mesmo quando estivesse fora de casa. E que fosse acess�vel por
uma tecnologia hands free. Compre j� o seu!

\begin{itemize}
	\item Expandir para v�rios aparelhos, tornando a beagle beagle um servidor
	centralizado no ambiente dom�stico. Isso tamb�m acarretar� em um n�mero maior
	e mais complexo de tabelas no banco de dados, contendo os diversos aparelhos
	e seus comandos.
	
	\item Em cada compartimento onde houvesse um aparelho eletr�nico a ser
	controlado, haveria um microcontrolador (a ser avaliado, preferencialmente 
	mais barato que o arduino) capaz de controlar determinado(s) aparelhos

	\item A beagle beagle e todos os outros microcontroladores estariam
	conectados � mesma rede LAN. Somente a beagle beagle precisaria estar
	conectada � internet, de modo que n�o houvesse limita��o de dist�ncia para a
	conex�o com o smartphone.
	
	\item Utilizar de forma eficiente o servidor Apache com o PHP, para que uma
	p�gina web seja criada e por meio dela informa��es de an�lise do sistema possam
	ser guardadas e acessadas remotamente pelo desenvolvedor, para que haja um
	algoritmo de detec��o e an�lise de problemas mais eficiente ao usu�rio. Essa nova
	funcionalidade tamb�m poder� disponibilzar um registro das atividades do usu�rio
	com o sistema, podendo ficar dispon�vel pelo usu�rio para monitoramento em
	\textit{high-level} ou para que o usu�rio at� mesmo possa ligar aparelhos remotamente.
\end{itemize}
	
\end{section}
%%% EOF %%%

\begin{section}{Trabalhos Futuros}
De acordo com o relato dispon�vel em~\cite{nielsen04}: ``Dado que o meu home
theater � modesto, ele requer que eu consiga manejar `apenas' 6 controles
remotos para a simples tarefa de assistir a um filme''. Seria �timo se houvesse
um controle remoto universal que permitisse acesso � \emph{todos} os aparelhos 
do ambiente residencial, mesmo os que est�o em c�modos diferentes. Ter o controle
sempre consigo e poder us�-lo por uma tecnologia \textit{hands free} mesmo
quando estivesse fora de casa � um sonho de qualquer consumidor.

\begin{itemize}
	\item Expandir para v�rios aparelhos, tornando a BeagleBone um servidor
	centralizado no ambiente dom�stico. Isso tamb�m acarretar� em um n�mero 
	maior e mais complexo de tabelas no banco de dados, contendo os diversos 
	aparelhos e seus comandos;
	
	\item Em cada compartimento onde houvesse um aparelho eletr�nico a ser
	controlado, haveria um microcontrolador (a ser avaliado, preferencialmente 
	mais barato que o Arduino) capaz de controlar determinado(s) aparelhos;

	\item A BeagleBone e todos os outros microcontroladores estariam
	conectados � mesma rede LAN. Somente a BBB precisaria estar conectada � 
	Internet, de modo que n�o houvesse limita��o de dist�ncia para a conex�o 
	com o \textit{smartphone};
	
	\item Utilizar de forma eficiente o servidor Apache com o PHP, para que uma
	p�gina web seja criada e por meio dela informa��es de an�lise do sistema 
	possam ser guardadas e acessadas remotamente pelo desenvolvedor, para que 
	haja um algoritmo de detec��o e an�lise de problemas mais eficiente ao 
	usu�rio. Essa nova funcionalidade tamb�m poder� disponibilzar um registro 
	das atividades do usu�rio com o sistema, podendo ficar dispon�vel para
	monitoramento em \textit{high-level} ou para que os aparelhos possam ser
	controlados remotamente a uma dist�ncia maior do que a limitada pela LAN.
\end{itemize}

\end{section}
%%% EOF %%%

\begin{section}{Dificuldades e Solu��es}
\begin{enumerate}

\item A BeagleBone n�o possui sa�da de �udio nativa, tampouco conectores do tipo
audio jack. No Debian, a sa�da padr�o de �udio � pelo conector HDMI, mas pode
ser substitu�da pelo USB mediante modifica��es em par�metros do kernel. O modo
mais f�cil, considerando que redirecionar a sa�da do eSpeak para um GPIO seria
muito trabalhoso, seria conectar um auto-falante USB � BeagleBone. Em se
tratando de um prot�tipo, um headphone faz o papel de um speaker que deve
consumir pouca energia e ter um tamanho limitado. Foi-se cogitada a constru��o
de um circuito com um amplificador LM386 para o speaker, por�m a obten��o de um
D/A PCM2707 n�o custaria menos de US\$ 15.

\item A BeagleBone n�o possui conex�o wifi. Al�m da dificuldade em atualizar o
kernel pra receber um shield/case, o mesmo teria de ser conectado na porta USB,
a qual j� estaria sendo usada pelo auto-falante. Portanto, a solu��o mais f�cil
foi conectar um roteador wifi � porta Ethernet da BBB atrav�s de um cabo RJ-45.

\item O Angstrom � PODRE. (Gabriel)

\item O c�digo do Arduino que hackeia as paradas � PODRE. (Thiago)

\item Acessar MySQL from C � PODRE (Pedro)
\end{enumerate}
\end{section}
%%% EOF %%%

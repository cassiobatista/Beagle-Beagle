\begin{section}{Dificuldades e Solu��es}
\begin{enumerate}

\item A BeagleBone n�o possui sa�da de �udio nativa, tampouco conectores do tipo
audio jack. No Debian, a sa�da padr�o de �udio � pelo conector HDMI, mas pode
ser substitu�da pelo USB mediante modifica��es em par�metros do kernel. O modo
mais f�cil, considerando que redirecionar a sa�da do eSpeak para um GPIO seria
muito trabalhoso, seria conectar um auto-falante USB � BeagleBone. Em se
tratando de um prot�tipo, um headphone faz o papel de um speaker que deve
consumir pouca energia e ter um tamanho limitado. Foi-se cogitada a constru��o
de um circuito com um amplificador LM386 para o speaker, por�m a obten��o de um
D/A PCM2707 n�o custaria menos de US\$ 15.

\item A BeagleBone n�o possui conex�o wifi. Al�m da dificuldade em atualizar o
kernel pra receber um shield/case, o mesmo teria de ser conectado na porta USB,
a qual j� estaria sendo usada pelo auto-falante. Portanto, a solu��o mais f�cil
foi conectar um roteador wifi � porta Ethernet da BBB atrav�s de um cabo RJ-45.

\item O Angstrom � PODRE. (Gabriel)

\item O c�digo do Arduino que hackeia as paradas � PODRE. (Thiago)

\item Outra dificuldade consistiu no fato de que para acessarmos o MySQL atrav�s da linguagem C precisamos passar a 
sintaxe correta para a fun��o em C que faz a query (\textit{$mysql\_query()$}), para isto ocorrer a sintaxe passada 
deve ser uma string bem espec\'{i}fica contendo o comando para o MySQL. Entretanto, a fim de automatizarmos essas 
queries, devemos manipular constantemente essas strings contendo a sintaxe, por exemplo, caso queiramos mudar o valor 
de um atributo em uma tabela espec\'{i}fica, apenas passamos o valor, o atributo e a tabela. Com isso, temos que usar 
diversas vezes aloca�\~{o}es de mem\'{o}ria (\textit{*malloc()}) e \textit{sprintf()}s, tornando o c\'{o}digo, em 
mais espec\'{i}fico o gerenciamento dos campos da mem\'{o}ria, ca\'{o}tico, apresentando algumas vezes falha de 
segmenta�\~{a}o. Posteriormente, com a adi�\~{a}o de outros aparelhos e mais tabelas, estas queries ficaram cada vez 
mais recorrentes e mais falhas poderam acontecer. 

Al\'{e}m disso, por estarmos acessando o MySQL por meio de uma outra linguagem (no caso C), n\~{a}o podemos prever 
agora poss\'{i}veis custos computacionais que essas diversas e constantes queries poder\~{a}o gerar nas threads 
dentro do BeagleBone Black\textregistered. Numa poss\'{i}vel e vindoura utiliza�\~{a}o de um servidor Apache com o 
PHP essa comunica�\~{a}o adicional e constante poder\'{a} resultar em significante custo computacional tornando-se um 
efeito \textit{snowball}, bola de neve.
\end{enumerate}
\end{section}
%%% EOF %%%
